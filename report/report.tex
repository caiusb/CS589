\documentclass[letterpaper]{article}

\title{Does the Version Control System affect commit understandability?}
\author{Mihai Codoban \and Caius Brindescu}
\date{}

\begin{document}

\maketitle

\section{Research questions}

\begin{enumerate}
	\item Do SVN commits take more time to understand than Git commits?
	\item Does user background (programming experience, industry experience etc) influence the time it takes to understand a commit?
	\item Does user background influence the quality and depth of the understanding of a commit or change description?
	\item Is there a common leave of change description between people?
\end{enumerate}

\section{Method}

\subsection{Participants}

Our participants are Java developers with Version Control Systems (VCS) experience. 
This includes:
\begin{itemize}
	\item at least 5 years of programming experience
	\item at least 1 year of experience using Java
	\item at least 1 year of experience using a VCS
\end{itemize}

Participants are recruited via email lists, university announcements (electronic and billboards), and local software companies targeting.

We will have 30 participants (5 people x 6 sessions). 
We plan on recruiting 47 people (30 participants, 2 no-shows per session and 5 pilots).

\subsection{Apparatus or Materials}

Participants will be viewing commits in a custom made software that for each commit shows the commit message, the diff (using the Eclipse Comparator Editor) and a text box where the commit description would be written.
They will have the following materials:
\begin{itemize}
	\item checklist of items to look for when understanding commits (embedded in commit window)
	\item commit information (commit message and commit diffs)
	\item questionnaire
\end {itemize}

Each participant will describe 3 SVN commits and 3 Git commits.
Each commit is chosen from a different repository.
Repositories are chosen from the top 6 Java repositories.
We selected 3 repositories using Git from GitHub and 3 repositories using SVN from SourceForge.
A commit is randomly chosen from the last commits in a project.
We believe this the last commits are representative of the change done in the later stages of the development.
To ensure the integrity of the commits we ignore commits that 
\begin{itemize}
	\item Contain changes to non-Java source files (build scripts, configuration files, etc);
	\item Are merge commits (this applies for Git only);
	\item Only change the formatting or comments.
\end{itemize}

\subsection{Procedure}

This will be a within subjects study. 
Each participant has to understand and describe 3 SVN commits and 3 Git commits. 
Their order in which we will apply the treatments will be random for each participant. 
This is done to eliminate the learning effect. 
The order will be determined by throwing a 6-sided die (or its programmatic equivalent).

The following script will be followed for each session:
\begin{enumerate}
	\item Informed Consent
	\item Administer tutorial
	\begin{enumerate}
		\item Explain the UI
		\item Explain the tasks to the participants
		\item Do a demo task
		\item Do a practice task
	\end{enumerate}
	\item Practice Task
	\item Tasks
	\begin{enumerate}
		\item Each task contains the commit message and the commit changes.
		\item The participant has to understand the commit. We provide a checklist of items to check.
		\item The participants are instructed to maximize detail
	\end{enumerate}
	\item Administer post session questionnaire
	\item Pay participants and receive receipts 
\end{enumerate}

\section{Data}

For each participant, we will collect the following:
\begin{enumerate}
	\item Time when they start a task
	\item Time when they complete a task
	\item The description of the commit
	\item Timestamp of all the key presses.
	\item Post-session questionnaire
\end{enumerate}

The time needed to understand a commit is determined by subtracting from the time it took to complete a task, the time it took the participants to type in their answer.
This will eliminate any variation due to varying typing speeds.

\section{Results}

\section{Discussion}

\subsection{Threats to validity}

\end{document}