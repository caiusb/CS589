\documentclass{beamer}

\mode<presentation>{ \usetheme{Szeged} }

\title{Does the version control tool affect the understandability of commits? \\ - Threats to validity - }
\author{Mihai Codoban \and Caius Brindescu} 
\date{March 5, 2014}

\begin{document}

% title frame
\begin{frame}
\titlepage
\end{frame}

%Description of the experiment (what are we looking for)
\begin{frame}
\frametitle{Research Questions}
\begin{enumerate}
	\item Do SVN commits take more time to understand than Git commits?
	\item Does user background influence the time it takes to understand a commit?
	\item Does user background influence the quality~/~depth of understanding a commit~/~change description?
	%should we remove this last one?
	\item Is there a common level of change description between people?
\end{enumerate}
\end{frame}

%What we did
\begin{frame}
\frametitle{Experiment description}
\begin{itemize}
	\item Within subject design
	\item Each participant had to describe the changes done in 6 commits (3 SVN + 3 Git)
	\item The commit order was randomized to avoid the learning effect
	\item Commits were randomly chosen for the top projects in SourceForge and GitHub
\end{itemize}
\end{frame}
\note{Describe the commit selection process}

%Internal threats: 
\begin{frame}
\frametitle{Internal threats}
\emph{Is there something inherent to how we collect and analyze the results that could skew our results?}

\begin{itemize}
	\item{\emph{Selection:}} The different experience levels of the participants can affect the results.
	\item{\emph{Maturation:}} The participants had to describe a total of 6 commits.
	Some reported that the tasks were getting boring after a while.
	\item{\emph{Definition of understanding:}} The participants had to self-assess when they understood all the changes. 
	This varies from person to person. 
	Offering a checklist and grading the responses can mitigate this risk.
\end{itemize}
\end{frame}

%External threats
\begin{frame}
\frametitle{External threats}
\emph{Are our results generalizable?}

\begin{itemize}
	\item{\emph{Commit selection:}} While the commits were selected randomly, there is a possibility that they are not representative for the general practice. 
	\item{\emph{Tool:}} For this study we build a special tool. 
	This might be different than the tools used in the wild.
	\item{\emph{Language:}} We only looked at Java projects.
	Results for other languages might be different.
\end{itemize}
\end{frame}

%Construct threats
\begin{frame}
\frametitle{Construct threats}

\emph{Are we asking the right questions?}

\begin{itemize}
	% should we sound our own horn here?
	\item Recent research has shown that committing behaviour is dependent on the VCS
	\item The main purpose (other than versioning), of VCS is to provide a context in which changes are made.
\end{itemize}
\end{frame}

%Reliability threats
\begin{frame}
\frametitle{Reliability threats}
\emph{Can others replicate our results?}

\begin{itemize}
	\item We will try out best to offer a complete and accurate description of the experiment.
	\item The source code of the tool used and the corpus will be made publicly available.
\end{itemize}
\end{frame}

\end{document}